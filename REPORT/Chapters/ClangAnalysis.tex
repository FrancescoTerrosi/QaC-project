\chapter{Clang Analysis}

\section{Introduction}

In this chapter it will be described the analysis process for all the tools used.\newline\newline
\textbf{Understand} is indeed the tools that gives the most accurate results in terms of checks, since it incorporates C/C++ MISRA standards, a beta version of the \textbf{CLang Static Analyzer}, which is a static analysis tool provided by the LLVM developers, and many other quality checks offered by SciTools itself.\newline
A simpler but also quite effective tool is \textbf{Cppcheck} which is designed to "provide unique code analysis to detect bugs and to focus on detecting undefined behaviour and dangerous coding constructs" \cite{bibitem2}. Also, as pointed by the developers, its main focus is to  "detect only real errors in the code (i.e. have very few false positives)". Cppcheck refers to the \textsl{Common Weakness Enumeration} standard for the analysis, a formal list of security issues published by the MITRE institute. It is also possible to check MISRA-C project compliance but it requires to buy the standard so this feature was not used. \newline
The last used tool is \textbf{flawfinder} which puts its focus more on security flaws rather than quality issues. This tool incorporates an option to run the analysis in order to detect possible false positives in an automated manner. This tools uses the CWE standard as Cppcheck does.\newline
Other tools such as \textbf{SonarQube} and \textbf{Cert C Rosechecker} were used but due to their characteristics they were unusable for our purpose.
\pagebreak

\section{Understand}

