\chapter{Conclusions}

In this work 5 different static analysis tools were used.\newline
The first difference between these tools is the set of rules used. Different standards have been covered:

\begin{itemize}
	\item MISRA-C++ and Code Quality practices - \textbf{Understand}
	\item MITRE-CWE - \textbf{Cppcheck}, \textbf{Flawfinder}
	\item CERT-C - \textbf{Rosechecker}
	\item Various standards and Code Quality practices - \textbf{SonarQube}
\end{itemize}

These tools are not meant to be mutually exclusive in their use. For example Cppcheck can be used in combination with Flawfinder to detect both \textsl{Code Quality} and \textsl{Security} issues, using the same standard (MITRE-CWE).\newline
Specific considerations must be made for the tool \textsl{Clang Static Analyzer} which was used as an Understand addon.\newline
The analysis with this tool provided no results. Seeing this we can suppose that the LLVM Developer Group defined a custom set of Quality rules without referring to any specific standard and that this set of rules was followed during the implementation of LLVM-Clang compiler and it is the one implemented in the Clang Static Analyzer tool.
\newline\newline

During the analyses it emerged that the file \textsl{CIndex.cpp}, which is the biggest source file in the analyzed directory \textsl{tools/libclang}, has always been the file with the most detected issues (always $\approx\: 50\%$ of the whole issues set). Since this anomaly was observed with all the tools we can consider it to be relevant and that this file needs some refactoring.\newline
Another interesting thing noticed is that if we exlclude limit cases such as \textsl{CIndex.cpp} or very small files (2-3 

ISSUES PROPORZIONATI A DIMENSIONE FILE
50\% DIVISO FRA GLI ALTRI 32 FILE OSSERVIAMO DISTRIBUZIONE PIU O MENO NORMALE (GAUSSIANA A CAMPANA)






IL PROGETTO FA SCHIFO QUALITY MERDA